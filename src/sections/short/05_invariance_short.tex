\section{Invariance}
\subsection{Invariance}
\subsubsection{Invariant Sets}
\ptitle{Positively Invariant Set}

A set $\mathcal{O}$ is said to be a positively invariant set for an autonomous system if
\begin{equation*}
    x(k) \in \mathcal{O} \Rightarrow x(k + 1) \in \mathcal{O}, \quad \forall k \in \{0, 1, \dots \}
\end{equation*}

\newpar{}
\ptitle{Maximal Positively Invariant Set $\mathcal{O}_\infty$}

The set $\mathcal{O}_\infty \subset \mathcal{X}$ is the maximal positively invariant set with respect to $\mathcal{X}$ if $\mathcal{O}_\infty$ is positively invariant and $\mathcal{O}_\infty$ contains all positively invariant sets.

\newpar{}
\ptitle{Geometric Condition for Invariance}

A set $\mathcal{O}$ is a positively invariant set if and only if
\begin{equation*}
    \mathcal{O} \subseteq \text{pre}(\mathcal{O})
\end{equation*}
\begin{equation*}
    \mathcal{O} \subseteq \text{pre}(\mathcal{O}) \iff \text{pre}(\mathcal{O})\cap \mathcal{O} = \mathcal{O}
\end{equation*}

\subsubsection{Pre-Sets}

Given a set $S$ and the dynamic system $x(k + 1) = g(x(k))$, the pre-set of $S$ is the set of states that evolve into the target set $S$ in \textbf{one} time step:
\begin{equation*}
    \text{pre}(S) := \{x \mid g(x) \in S \}
\end{equation*}

\ptitle{Linear Autonomous Systems}
\begin{equation*}
    \text{pre}(S) = \{x \mid F A x \leq f \}
\end{equation*}

\subsubsection{Computing Invariant Sets}

\begin{algorithmic}
    \State{} \textbf{Input:} $g$, $X$
    \State{} \textbf{Output:} $\mathcal{O}_\infty$
    \State{} $\Omega_0 \gets X$
    \While{true}
    \State{} $\Omega_{i+1} \gets \text{pre}(\Omega_i) \cap \Omega_i$
    \If{$\Omega_{i+1} = \Omega_i$}
    \State{} \textbf{return} $\mathcal{O}_\infty = \Omega_i$
    \EndIf{}
    \EndWhile{}
\end{algorithmic}

\subsection{Control Invariance}
\subsubsection{Control Invariant Sets}
A set $\mathcal{C} \subseteq \mathcal{X}$ is said to be a control invariant set if
\begin{gather*}
    x(k) \in \mathcal{C} \Rightarrow \exists u(k) \in \mathcal{U}\\
    \text{such that } g(x(k), u(k)) \in \mathcal{C} \text{ for all } k \in \mathbb{N}^+
\end{gather*}

\newpar{}
\ptitle{Maximal Control Invariant Set $\mathcal{C}_\infty$}

The set $\mathcal{C}_\infty$ is said to be the maximal control invariant set for the system $x(k + 1) = g(x(k), u(k))$ subject to the constraints $(x, u) \in \mathcal{X} \times \mathcal{U}$ if it is control invariant and \textbf{contains all control invariant sets} contained in $\mathcal{X}$.


\subsubsection{Conceptual Calculation of Control Invariant Sets}

The concept of a pre-set extends to systems with exogenous inputs:
\begin{equation*}
    \text{pre}(S) := \{x \mid \exists u \in \mathcal{U} \text{ s.t. } g(x, u) \in S \}
\end{equation*}
A set $\mathcal{C}$ is a control invariant set if and only if $\mathcal{C} \subseteq \text{pre}(\mathcal{C})$.

\newpar{}
\ptitle{Pre-set Computation for Constrained LTI System}
\begin{align*}
    \text{pre}(S) & = \{x \mid \exists u \in \mathcal{U}, \; Ax + Bu \in S \}                                                                                                            \\
                  & = \{x \mid \exists u \in \mathcal{U}, \; F A x + F B u \leq f \}                                                                                                     \\
                  & = \left\{ x \mid \exists u, \begin{bmatrix} F A & F B \\ 0 & G \end{bmatrix} \begin{bmatrix} x \\ u \end{bmatrix} \leq \begin{bmatrix} f \\ g \end{bmatrix} \right\}
\end{align*}

\paragraph{Usefulness}
MPC implicitly describes a (suboptimal) control invariant set such that it's easy to represent and compute.

\paragraph{Control Laws from Control Invariant Sets}

We can synthesize a control law from a control invariant set by solving the following \textbf{optimization problem}:
\begin{equation*}
    \kappa(x) := \arg\min \{ f(x, u) \mid g(x, u) \in \mathcal{C},\; u \in \mathcal{U} \},
\end{equation*}
that will satisfy constraints (but not necessarily converges).

\subsection{Computation of Simple Invariant Sets}

\subsubsection{Polytopes}
\paragraph{Intersection}
\noindent
\begin{equation*}
    S \cap T = \left\{x \Bigg|\begin{bmatrix}
        C \\D
    \end{bmatrix}x \leq \begin{bmatrix}
        c \\d
    \end{bmatrix}\right\}
\end{equation*}

\paragraph{Equality Test (Subset Test)}
$P=\{x\mid Cx\leq c\}$ is a subset of $Q=\{x\mid Dx\leq d\}$ if for each row, the \textbf{support function} is a subset of $D$:
\begin{equation*}
    h_P(D_i) \leq d_i
\end{equation*}
where the support (extremum of $P$ in direction $D_i$) is
\begin{align*}
    h_p(D_i) :=     & \max_x D_i x \\
    \mathrm{s.t.}\; & Cx\leq c
\end{align*}
% \begin{center}
%     \includegraphics[width = 0.6\linewidth]{05_support_fnc.png}
% \end{center}

\subsubsection{Ellipsoids}
If $V:\mathbb{R}^d\to\mathbb{R}$ is a Lyapunov function for the system $x(k+1)=g(x(k))$, then the sublevel set
\begin{equation*}
    Y = \{x\mid V(x)\leq \alpha\}
\end{equation*}
is an invariant set for all $\alpha \geq 0$.

\paragraph{Linear Systems}
\noindent
\begin{equation*}
    Y_\alpha = \left\{x\mid x^\top Px\leq \alpha\right\} \subset \mathcal{X}= \left\{x\mid Fx\leq f\right\}
\end{equation*}

\ptitle{Maxmium Invariant Set}
\noindent
\begin{gather*}
    \max_{\alpha}\alpha \\
    \text{s.t. }h_{Y_{\alpha}}(F_{i})\leq f_{i} \quad \forall i\in\{1,...,n\}
\end{gather*}
\begin{equation*}
    h_{Y_\alpha}(F_i)=\|P^{-1/2}F_i^T\|\sqrt{\alpha}
\end{equation*}
\noindent
\begin{equation*}
    \alpha^* = \min_{i\in \{1,\ldots, n\}} \frac{f_i^2}{F_i P^{-1} F_i^\top}
\end{equation*}