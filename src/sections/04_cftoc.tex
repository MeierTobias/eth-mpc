\section{Constraint Finite Time Optimal Control (CFTOC)}

\subsection{Constraint Linear Optimal Control}

For a linear model the CFTOC problem is given by
\begin{align*}
    J^*(x(k)) =            & \min_U \: I_f(x_N) + \sum_{i=0}^{N-1}I(x_i,u_i)                    \\
    \text{subject to}\quad & x_{i+1} = Ax_i +Bu_i, \quad i = 0, \ldots, N-1                     \\
                           & x_i \in \mathcal{X}, u_i \in \mathcal{U}, \quad i = 0, \ldots, N-1 \\
                           & x_N \in \mathcal{X}_f                                              \\
                           & x_0 = x(k)
\end{align*}
where $I(x_i,u_i)$ is the stage cost, $I_f(x_N)$ represents an approximation of the tail cost and $\mathcal{X}_f$ of the tail constraints, respectively. $N$ is the time horizon and $\mathcal{X,U,X}_f$ are polyhedral regions.

\newpar{}
\ptitle{Feasible Set}

The feasible set describes a set of states for which the optimal control problem is feasible:
\begin{align*}
    \mathcal{X}_N = & \left\{ x_0 \in \mathbb{R}^n | \exists\left(u_0, \ldots, u_{N-1}\right) \text{such that } x_i\in\mathcal{X}, u_i\in\mathcal{U} \right. \\
                    & \left.i=0,\ldots, N-1, x_N\in\mathcal{X}_f, \text{where } x_{i+1}=Ax_i + Bu_i\right\}
\end{align*}

Or in other words, the feasible set describes initial conditions from which it is possible to get into the final set $\mathcal{X}_f$ within $N-1$ steps.

\subsubsection{Quadratic Cost}

The cost function that incorporates a squared euclidean norm is composed of
\begin{align*}
    I_f(x_N)   & = x_N^T P x_N                     \\
    I(x_i,U_i) & = x_i^\top Q x_i + u_i^\top R u_i
\end{align*}
with $P\succeq0, Q\succeq0, R\succ0$.

\newpar{}

The quadratic cost CFTOC problem
\begin{align*}
    J^*(x(k)) =            & \min_U \: x_N^T P x_N + \sum_{i=0}^{N-1}x_i^\top Q x_i + u_i^\top R u_i \\
    \text{subject to}\quad & x_{i+1} = Ax_i +Bu_i, \quad i = 0, \ldots, N-1                          \\
                           & x_i \in \mathcal{X}, u_i \in \mathcal{U}, \quad i = 0, \ldots, N-1      \\
                           & x_N \in \mathcal{X}_f                                                   \\
                           & x_0 = x(k)
\end{align*}

can be transformed into a QP problem
\begin{align*}
    \min_{z\in\mathbb{R}^n} & \frac{1}{2}z^\top Hz + q^\top z + r \\
    \text{subject to}\quad  & Gz \leq h                           \\
                            & Az = b
\end{align*}

either with (\ref{cftoc_QP_with_subs}) or without (\ref{cftoc_QP_without_subs}) substitution of the future states $X$.

\paragraph{Construction of QP without Substitution}\label{cftoc_QP_without_subs}

This approach keeps the state equations as equality constraints. The CFTOC problem is transformed into the QP problem
\begin{align*}
    J^*(x(k)) =            & \min_{z}\begin{bmatrix}
                                         z^\top & x(t)
                                     \end{bmatrix}
    \begin{bmatrix}
        \bar{H} & 0 \\
        0       & Q
    \end{bmatrix}
    \begin{bmatrix}
        z^\top & {x(k)}^\top
    \end{bmatrix}^\top                                       \\
    \text{subject to}\quad & G_{in} z \leq w_{in} +E_{in} x(k) \\
                           & G_{eq} z = E_{eq} x(k)
\end{align*}
where
\begin{equation*}
    z = \begin{bmatrix}
        x_1^\top & \cdots & x_N^\top & u_0^\top & \cdots & u_{N-1}^\top
    \end{bmatrix}^\top
\end{equation*}
the equalities are given from the system dynamics
\begin{equation*}
    x_{i+1} = Ax_i + Bu_i
\end{equation*}
resulting in
\begin{gather*}
    G_{eq} = \left[
        \begin{array}{cccc|cccc} % ChkTex 44
            I  &   &        &   & -B &    &        &    \\
            -A & I &        &   &    & -B &        &    \\
               &   & \ddots &   &    &    & \ddots &    \\
               &   & -A     & I &    &    &        & -B
        \end{array}
        \right] \\
    E_{eq} = \begin{bmatrix}
        A & 0 & \cdots & 0
    \end{bmatrix}^\top
\end{gather*}

The inequalities
\begin{equation*}
    G_{in} z \leq w_{in} + E_{in} x(k)
\end{equation*}
given by
\begin{gather*}
    \mathcal{X} = \{x|A_x x \leq b_x\} \\
    \mathcal{U} = \{u|A_u u \leq b_u\} \\
    \mathcal{X}_f = \{x|A_f x \leq b_f\}
\end{gather*}
are constructed as
\begin{gather*}
    G_{in} = \left[
        \begin{array}{cccc|cccc} % ChkTex 44
            0   &        &     &     & 0   &        &        &     \\
            \hline % ChkTex 44
            A_x &        &     &     & 0   &        &        &     \\
                & \ddots &     &     &     & \ddots &        &     \\
                &        & A_x &     &     &        & 0      &     \\
                &        &     & A_f &     &        &        & 0   \\
            \hline % ChkTex 44
            0   &        &     &     & A_u &        &        &     \\
                & \ddots &     &     &     & A_u    &        &     \\
                &        & 0   &     &     &        & \ddots &     \\
                &        &     & 0   &     &        &        & A_u
        \end{array}
        \right]\\
    w_{in} = \left[
        \begin{array}{ccccc|ccccc} % ChkTex 44
            b_x & b_x & \cdots & b_x & b_f & b_u & b_u & \cdots & b_u & b_u
        \end{array}^\top
        \right]\\
    E_{in} =\begin{bmatrix}
        -A_x^\top & 0 & \cdots & 0
    \end{bmatrix}^\top
\end{gather*}
and finally the cost matrix $\bar{H}$ is
\begin{equation*}
    \bar{H} = \left[
        \begin{array}{cccc|cccc} % ChkTex 44
            Q &        &   &   &   &        &   \\
              & \ddots &   &   &   &        &   \\
              &        & Q &   &   &        &   \\
              &        &   & P &   &        &   \\
            \hline % ChkTex 44
              &        &   &   & R &        &   \\
              &        &   &   &   & \ddots &   \\
              &        &   &   &   &        & R
        \end{array}
        \right]
\end{equation*}

This can be constructed using the Matlab function

\begin{small}
    \texttt{barH = blkdiag(kron(eye(N-1),Q),P,kron(eye(N),R))}
\end{small}

\paragraph{Construction of QP with Substitution}\label{cftoc_QP_with_subs}

The future states are only dependent on the current state and the applied input sequence, hence they can be uniquely constructed from these and can therefore be substituted. To do so the same method as for the LQR batch approach (\ref{unconst_lqr}) is used:
\begin{align*}
    \underbrace{\begin{bmatrix}
                        x_1    \\
                        x_2    \\
                        \vdots \\
                        x_N
                    \end{bmatrix}}_{X} = &
    \underbrace{\begin{bmatrix}
                        A      \\
                        A^2    \\
                        \vdots \\
                        A^N
                    \end{bmatrix}}_{S^x} x(k) +
    \underbrace{\begin{bmatrix}
                        B        & 0      & \cdots & 0 \\
                        AB       & B      & \cdots & 0 \\
                        \vdots   & \ddots & \ddots & 0 \\
                        A^{N-1}B & \cdots & AB     & B
                    \end{bmatrix}}_{S^u}
    \underbrace{\begin{bmatrix}
                        u_0    \\
                        u_1    \\
                        \vdots \\
                        u_{N-1}
                    \end{bmatrix}}_{U}                 \\
    X =                         & S^x x(k) + S^u U
\end{align*}
By substituting the $X$ vector in the cost function of~\ref{cftoc_QP_without_subs} with this formulation, one can rewrite the cost only depending on $U$ and $x(k)$
\begin{align*}
    J(x(k), U) & = U^\top HU + 2{x(k)}^\top FU + {x(k)}^\top Y x(k) \\
               & = \begin{bmatrix}
                       U^\top & {x(k)}^\top
                   \end{bmatrix}
    \begin{bmatrix}
        H & F^\top \\
        F & Y
    \end{bmatrix}
    {\begin{bmatrix}
         U^\top & {x(k)}^\top
     \end{bmatrix}}^\top
\end{align*}
where $\begin{bmatrix}H&F^\top\\F&Y\end{bmatrix}\succeq0$ since $J(x(k),U) \geq 0$ by assumption.
\newpar{}
The inequality constraints can be rewritten as
\begin{equation*}
    GU\leq w + Ex(k)
\end{equation*}
for
\begin{gather*}
    \mathcal{X} = \{x|A_x x \leq b_x\} \\
    \mathcal{U} = \{u|A_u u \leq b_u\} \\
    \mathcal{X}_f = \{x|A_f x \leq b_f\}
\end{gather*}
with
\begin{gather*}
    G = \begin{bmatrix}
        A_u           & 0             & \cdots & 0      \\
        0             & A_u           & \cdots & 0      \\
        \vdots        & \vdots        & \ddots & \vdots \\
        0             & 0             & \cdots & A_u    \\
        0             & 0             & \cdots & 0      \\
        A_x B         & 0             & \cdots & 0      \\
        A_x AB        & A_x B         & \cdots & 0      \\
        \vdots        & \vdots        & \ddots & \vdots \\
        A_f A^{N-1} B & A_f A^{N-2} B & \cdots & A_f B
    \end{bmatrix}\\
    E = \begin{bmatrix}
        0        \\
        0        \\
        \vdots   \\
        0        \\
        -A_x     \\
        -A_x A   \\
        -A_x A^2 \\
        \vdots   \\
        -A_f A^N
    \end{bmatrix}, \quad
    w = \begin{bmatrix}
        b_u    \\
        b_u    \\
        \vdots \\
        b_u    \\
        b_x    \\
        b_x    \\
        b_x    \\
        \vdots \\
        b_f
    \end{bmatrix}
\end{gather*}

With this substitutions the final QP problem becomes
\begin{align*}
    J^*(x(k), U) =         & \min_{U}\begin{bmatrix}
                                         U^\top & {x(k)}^\top
                                     \end{bmatrix}
    \begin{bmatrix}
        H & F^\top \\
        F & Y
    \end{bmatrix}
    {\begin{bmatrix}
         U^\top & {x(k)}^\top
     \end{bmatrix}}^\top                                 \\
    \text{subject to}\quad & GU \leq w + Ex(k)
\end{align*}

\paragraph{State Feedback Solution}

%TODO: Lecture 5 

\subsubsection[1-Norm and Inf-Norm Cost]{1-Norm and $\infty$-Norm Cost}

The cost function that incorporates the $p=1$ or $p=\infty$ norm is composed of
\begin{align*}
    I_f(x_N)   & = {\lVert Px_N \rVert}_p                          \\
    I(x_i,U_i) & = {\lVert Qx_i \rVert}_p + {\lVert Ru_i \rVert}_p
\end{align*}
with $P,Q,R$ having full column rank.

%TODO: Lecture 5 

\paragraph{Construction of LP with Substitution}

%TODO: Lecture 5 

\paragraph{State Feedback Solution}

%TODO: Lecture 5 