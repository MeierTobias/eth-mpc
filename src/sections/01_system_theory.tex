\section{System Theory}

\subsection{Models of Dynamic Systems}

The basic formulation of a nonlinear, time-invariant, continues-time, state space model can be stated as
\begin{align*}
    \dot{x} & = g(x,u) \\
    y       & = h(x,u)
\end{align*}
with
\begin{align*}
    x & \in \mathbb{R}^n                                    &  & \text{state vector}    \\
    u & \in \mathbb{R}^m                                    &  & \text{input vector}    \\
    y & \in \mathbb{R}^p                                    &  & \text{output vector}   \\
    g & : \mathbb{R}^n \times \mathbb{R}^m \to \mathbb{R}^n &  & \text{system dynamics} \\
    h & : \mathbb{R}^n \times \mathbb{R}^m \to \mathbb{R}^p &  & \text{output function}
\end{align*}


\subsection{Linearization}

The first order \textbf{Taylor Expansion} around an operating point $\bar{x}$ of $f(x)$ is given by
\begin{equation*}
    f(x) \approx f(\bar{x}) + \left. \frac{\partial f}{\partial x^\top} \right\rvert_{x=\bar{x}} (x-\bar{x})
\end{equation*}
with
\begin{equation*}
    \frac{\partial f}{\partial x^\top} = \begin{bmatrix}
        \frac{\partial f_1}{\partial x_1} & \frac{\partial f_1}{\partial x_2} & \cdots & \frac{\partial f_1}{\partial x_n} \\
        \vdots                            &                                   & \vdots                                     \\
        \frac{\partial f_n}{\partial x_1} & \frac{\partial f_n}{\partial x_2} & \cdots & \frac{\partial f_n}{\partial x_n}
    \end{bmatrix}
\end{equation*}
The linearized, time-invariant, continuous-time state space model around the stationary operating point $x_s, u_s$ can then be obtained with
\begin{align*}
    \dot{x}  & =
    \overbrace{\left. \frac{\partial g}{\partial x^\top} \right\rvert_{\substack{x_s  \\u_s}}}^{A^c \in \mathbb{R}^{n\times n}} \Delta x +
    \overbrace{\left. \frac{\partial g}{\partial u^\top} \right\rvert_{\substack{x_s  \\u_s}}}^{B^c \in \mathbb{R}^{n\times m}} \Delta u \\
    \Delta y & =
    \underbrace{\left. \frac{\partial h}{\partial x^\top} \right\rvert_{\substack{x_s \\u_s}}}_{C \in \mathbb{R}^{p\times n}} \Delta x +
    \underbrace{\left. \frac{\partial h}{\partial u^\top} \right\rvert_{\substack{x_s \\u_s}}}_{D \in \mathbb{R}^{p \times m}} \Delta u
\end{align*}
where
\begin{align*}
    \dot{x}_s & = g(x_s,u_s) = 0 \\
    y_s       & = h(x_s,u_s)
\end{align*}
and
\begin{align*}
    \Delta x       & = x - x_s                               \\
    \Delta u       & = u - u_s                               \\
    \Delta y       & = y - y_s                               \\
    \Delta \dot{x} & = \dot{x} - \underbrace{\dot{x}_s}_{=0}
\end{align*}

\ptitle{Exact Solution for LTI CT SS Models}

\begin{equation*}
    x(t) = e^{A^c(t-t_0)}x_0 + \int_{t_0}^{t}e^{A^c(t-\tau)}B^c u(\tau)d\tau
\end{equation*}
where
\begin{equation*}
    e^{A^c t} := \sum_{n=0}^{\infty}\frac{{(A^c t)}^n}{n!}
\end{equation*}

\subsection{Discretization}

\ptitle{Euler Discretization}
\begin{equation*}
    \dot{x}^c \approx \frac{x^c(t + T_s)-x^c(t)}{T_s}
\end{equation*}
with $T_s$ describing the sampling time, hence
\begin{align*}
    x(k) & := x^c(t_0 + kT_s) \\
    u(k) & := u^c(t_0 + kT_s)
\end{align*}
Then the DT model is given by
\begin{align*}
    x(k+1) & = x(k) + T_s g^c (x(k),u(k)) & = g(x(k),u(k)) \\
    y(k)   & = h^c(x(k),u(k))             & = h(x(k),u(k))
\end{align*}
Therefor a LTI system becomes
\begin{align*}
    x(k+1) & = \overbrace{\mathbb{I} + T_s A^c}^{=A^d} x(k) + \overbrace{T_s B^c}^{=B^d} u(k) \\
    y(k)   & = \underbrace{C^c}_{C^d} x(k) + \underbrace{D^c}_{D^d} u(k)
\end{align*}

\newpar{}
\ptitle{Exact Discretization of LTI}

If the input $u$ is held constant over a sampling interval (ZOH) one can retrieve an exact discretization of the LTI system
\begin{equation*}
    x(t_{k+1}) = \underbrace{e^{A^c T_s}}_{=A} x(t_k) + \underbrace{\int_{0}^{T_s} e^{A^c(T_s - \tau)}B^c d\tau}_{B} u(t_k)
\end{equation*}
with
\begin{equation*}
    B={(A^c)}^{-1}(A-\mathbb{I})B^c
\end{equation*}
if $A^c$ is invertible.

\newpar{}
\ptitle{Solution of DT LTI System}

If the input sequence $\{u(k), \ldots, u(k+N-1)\}$ and the initial state $x(k)$ is known the the solution for the discrete time system at time $k+N$ is given by
\begin{equation*}
    x(k+N) = A^N x(k) + \sum_{i=0}^{N-1} A^i B u(k+N-1-i)
\end{equation*}


\subsection{LTI DT System Analysis}

This section only summarizes the most important properties. Further details can be found in {\small\url{https://github.com/MeierTobias/eth-control-systems-2}}.

\subsubsection{Coordinate Transformastions}

\begin{align*}
    \tilde{x}(k)   & = Tx(k); \qquad \det(T) \neq 0                                                    \\
    \tilde{x}(k+1) & = \overbrace{TAT^{-1}}^{\tilde{A}} \tilde{x}(k) + \overbrace{TB}^{\tilde{B}} u(k) \\
    y(k)           & = \underbrace{CT^{-1}}_{\tilde{C}} \tilde{x}(k) + \underbrace{D}_{\tilde{D}}u(k)
\end{align*}

\subsubsection{Stability}
The DT LTI System
\begin{equation*}
    x(k+1) = Ax(k)
\end{equation*}
is globally asymptotically stable iff
\begin{equation*}
    \lvert \lambda_j \rvert < 1
\end{equation*}
for all eigenvalues $\lambda_j$ of $A$.

\subsubsection{Reachability and Observability}
\noindent\begin{equation*}
    \mathcal{R} \subseteq \mathcal{C}, \quad \mathcal{R} \subset\mathcal{S}, \quad \mathcal{O} \subset \mathcal{D}
\end{equation*}
% Please ignore the mess I've created here, I don't feel comfortable either.
\begin{center}
    \begin{tikzcd}[ampersand replacement=\&]
        \substack{\text{unstable} \\ \text{modes} \\ \in{} \mathcal{R} \\ \in{} \mathcal{O}}      \&[2em] \substack{\text{all} \\ \text{modes}} \&[-3em] \\[-2em]
        \& \mathcal{C} \arrow[d, teal, xshift=-0.5ex,
        "{\color{teal}\begin{matrix}                        %chktex 18
            \mathsf{CT}: & \text{always}                \\
            \mathsf{DT}: & \mathbf{A}_d \text{ invert.}
        \end{matrix}}
        "'                                                  %chktex 18
        ] \& {\footnotesize\begin{cases}
                    \mathsf{CT:} & \mathbf{x}_0 = 0 \to~\mathbf{x_f} \\
                    \mathsf{DT:} & \mathbf{x}_0 = \mathbf{x_f} \to~0
                \end{cases}}
        \\[2em]
        \mathcal{S}                                     \& \mathcal{R} \arrow[l]\arrow[u, xshift=0.5ex] \&
        {\footnotesize\begin{cases}
                    \mathbf{x}_0 = 0 \to~\mathbf{x_f}
                \end{cases}}
        \\
        \mathcal{D}                                     \& \mathcal{O} \arrow[l]    \&
        {\footnotesize\begin{cases}
                    \mathbf{x}_0 \gets~\mathbf{u},\mathbf{y}
                \end{cases}}
    \end{tikzcd}
\end{center}

\subsubsection{Controllability}

A system is controllable if for any pair of states $x(0), x^*$ there exists a finite time N and a control sequence $u$ such that $x(N)=x^*$

\newpar{}
The \textbf{controllability matrix}
\begin{equation*}
    \mathcal{C} :=
    \begin{bmatrix}
        B & AB & \cdots & A^{n-1} B
    \end{bmatrix}
\end{equation*}
has to have \textbf{full row rank}.
\newpar{}
\textbf{Stabilizability} iff all uncontrollable modes are stable
\begin{itemize}
    \item A system is stabilizable if all unstable modes are reachable.
    \item Reachability always implies stabilizability
\end{itemize}

\subsubsection{Observability}


A system is observable if there exists a finite $N$ such that for every $x(0)$ the measurements $y(0), \ldots, y(N-1)$ uniquely distinguish the initial state $x(0)$.
\newpar{}
The \textbf{observability matrix}
\begin{equation*}
    \mathcal{O} :=
    \begin{bmatrix}
        C      \\
        CA     \\
        CA^2   \\
        \vdots \\
        CA^{n-1}
    \end{bmatrix}
\end{equation*}
has to have \textbf{full column rank}.
\newpar{}
\textbf{Detectability}
\begin{itemize}
    \item A system is detectable if all unstable modes are observable.
    \item Observability always implies detectability
\end{itemize}


\subsection{Nonlinear System Analysis}

\subsubsection{Lyapunov Stability}

The equilibrium point $\bar{x}$ of a DT system is \textbf{Lyapunov stable} iff for every $\epsilon >0$ there exists a $\delta(\epsilon)$ such that
\begin{equation*}
    \lVert x(0) - \bar{x} \rVert < \delta(\epsilon) \to \lVert x(k) - \bar{x} \rVert < \epsilon, \; \forall k \geq 0
\end{equation*}

\begin{center}
    \includegraphics[width=0.3\linewidth]{01_Lyapunov_stability.png}
\end{center}

\ptitle{Global Asymptotic Stability}

An equilibrium point $\bar{x}$ of a system is globally asymptotically stable if it is Lyapunov stable and attractive

\begin{equation*}
    \lim_{k\to\infty} \lVert x(k) - \bar{x} \rVert = 0, \; \forall x(0) \in \mathbb{R}
\end{equation*}


\subsubsection{Global Lyapunov Function}

Consider the equilibrium point $\bar{x}=0$. A function $V:\mathbb{R}^n\to \mathbb{R}$, continuous at the origin, finite $\forall x \in \mathbb{R}^n$, and such that
\begin{gather*}
    \lVert x \rVert \to \infty \Rightarrow V(x) \to \infty\\
    V(0)=0 \text{ and } V(x)>0, \quad \forall x \in \mathbb{R}^n \setminus\{0\}\\
    V(g(x)) - V(x) \leq -\alpha(x) \quad \forall x \in \mathbb{R}^n
\end{gather*}
where $\alpha:\mathbb{R}^n\to \mathbb{R}$ is continuous positive definite, is called a \textbf{Lyapunov function}.

\subsubsection{Global Lyapunov Stability}

If a system admits a Lyapunov function $V(x)$, then $x=0$ is \textbf{globally asymptotically stable}.
\newpar{}
If $V(x)$ satisfies the conditions only with $\alpha(x)$ being positive \textbf{semi}definite, then $x=0$ is \textbf{globally Lyapunov stable}.

\subsection{Global Lyapunov Stability of LTI DT System}

For the LTI DT system
\begin{equation*}
    x(k+1)=Ax(k)
\end{equation*}
One can take
\begin{equation*}
    V(x) = x^\top Px
\end{equation*}
as a candidate Lyapunov function with $P>0$ (positive definite). It's easy to see that it satisfies
\begin{gather*}
    V(0) = 0 \text{ and } V(x) > 0 \\
    \lVert x \rVert \to \infty \Rightarrow V(x) \to \infty
\end{gather*}
to construct the $\alpha(x)$ function the ``energy decrease'' condition is used
\begin{align*}
    V(Ax) - V(x) & = x^\top A^\top PAx - x^\top Px                      \\
                 & = x^\top \left(A^\top PA - P\right)x \leq -\alpha(x)
\end{align*}
If $\alpha(x)$ is chosen as
\begin{equation*}
    \alpha(x) = x^\top Qx, \quad Q > 0
\end{equation*}
the condition can be satisfied if a $P>0$ can be fouhnd that solves the \textbf{discrete-time Lyapunov equation}
\begin{equation*}
    A^\top PA -P = -Q, \quad Q > 0
\end{equation*}

The discrete-time Lyapunov equation has a unique solution $P > 0$ iff $A$ has all eigenvalues inside the unit circle, i.e.\ iff the system $x(k + 1) = Ax(k)$ is stable.

%Double check if this is correct:
\newpar{}
\ptitle{Closed Loop Control}
When using LQR, the matrix $P$ that satisfies the DARE equation also satisfies the Lyapunov equation
\noindent\begin{equation*}
    {(A+BK)}^\top P{(A+BK)} -P = -Q, \quad Q>0
\end{equation*}

\begin{examplesection}[Example: Lyapunov Stability for Linear Systems]
    Given the LTI DT system
    \begin{equation*}
        x(k+1) = \begin{bmatrix}
            1 & 1 \\
            0 & 1
        \end{bmatrix} x(k) +
        \begin{bmatrix}
            1 \\
            0.5
        \end{bmatrix} u(k)
    \end{equation*}
    with
    \begin{equation*}
        u(k) = Kx(k) = -\begin{bmatrix}
            0.1160 & 0.5385
        \end{bmatrix} x(k)
    \end{equation*}
    We choose $Q=\mathbb{I}$ and solve the discrete time Lyapunov equation
    \begin{gather*}
        {\left(A+BK\right)}^\top P \left(A+BK\right) - P  = -Q \\
        P = \begin{bmatrix}
            3.3921 & 2.7757 \\
            2.7757 & 7.7136
        \end{bmatrix}
    \end{gather*}
    Since $V(x)=x^\top Px$ is a quadratic function, all level sets are ellipsoids with shape matrix $P$ and the closed loop evolution can be simulated as shown.
    \begin{center}
        \includegraphics[width=\linewidth]{01_LTI_DT_Lyapunov_example.png}
    \end{center}
\end{examplesection}