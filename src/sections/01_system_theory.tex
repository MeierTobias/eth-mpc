\section{System Theory}

\subsection{Models of Dynamic Systems}

The basic formulation of a nonlinear, time-invariant, continues-time, state space model can be stated as
\begin{align*}
    \dot{x} & = g(x,u) \\
    y       & = h(x,u)
\end{align*}
with
\begin{align*}
    x & \in \mathbb{R}^n                                    &  & \text{state vector}    \\
    u & \in \mathbb{R}^m                                    &  & \text{input vector}    \\
    y & \in \mathbb{R}^p                                    &  & \text{output vector}   \\
    g & : \mathbb{R}^n \times \mathbb{R}^m \to \mathbb{R}^n &  & \text{system dynamics} \\
    h & : \mathbb{R}^n \times \mathbb{R}^m \to \mathbb{R}^p &  & \text{output function}
\end{align*}


\subsection{Linearization}

The first order \textbf{Taylor Expansion} around an operating point $\bar{x}$ of $f(x)$ is given by
\begin{equation*}
    f(x) \approx f(\bar{x}) + \left. \frac{\partial f}{\partial x^\top} \right\rvert_{x=\bar{x}} (x-\bar{x})
\end{equation*}
with
\begin{equation*}
    \frac{\partial f}{\partial x^\top} = \begin{bmatrix}
        \frac{\partial f_1}{\partial x_1} & \frac{\partial f_1}{\partial x_2} & \cdots & \frac{\partial f_1}{\partial x_n} \\
        \vdots                            &                                   & \vdots                                     \\
        \frac{\partial f_n}{\partial x_1} & \frac{\partial f_n}{\partial x_2} & \cdots & \frac{\partial f_n}{\partial x_n}
    \end{bmatrix}
\end{equation*}
The linearized, time-invariant, continuous-time state space model around the stationary operating point $x_s, u_s$ can then be obtained with
\begin{align*}
    \dot{x}  & =
    \overbrace{\left. \frac{\partial g}{\partial x^\top} \right\rvert_{\substack{x_s  \\u_s}}}^{A^c \in \mathbb{R}^{n\times n}} \Delta x +
    \overbrace{\left. \frac{\partial g}{\partial u^\top} \right\rvert_{\substack{x_s  \\u_s}}}^{B^c \in \mathbb{R}^{n\times m}} \Delta u \\
    \Delta y & =
    \underbrace{\left. \frac{\partial h}{\partial x^\top} \right\rvert_{\substack{x_s \\u_s}}}_{C \in \mathbb{R}^{p\times n}} \Delta x +
    \underbrace{\left. \frac{\partial h}{\partial u^\top} \right\rvert_{\substack{x_s \\u_s}}}_{D \in \mathbb{R}^{p \times m}} \Delta u
\end{align*}
where
\begin{align*}
    \dot{x}_s & = g(x_s,u_s) = 0 \\
    y_s       & = h(x_s,u_s)
\end{align*}
and
\begin{align*}
    \Delta x       & = x - x_s                               \\
    \Delta u       & = u - u_s                               \\
    \Delta y       & = y - y_s                               \\
    \Delta \dot{x} & = \dot{x} - \underbrace{\dot{x}_s}_{=0}
\end{align*}

\ptitle{Exact Solution for LTI CT SS Models}

\begin{equation*}
    x(t) = e^{A^c(t-t_0)}x_0 + \int_{t_0}^{t}e^{A^c(t-\tau)}B^c u(\tau)d\tau
\end{equation*}
where
\begin{equation*}
    e^{A^c t} := \sum_{n=0}^{\infty}\frac{{(A^c t)}^n}{n!}
\end{equation*}

\subsection{Discretization}

\ptitle{Euler Discretization}
\begin{equation*}
    \dot{x}^c \approx \frac{x^c(t + T_s)-x^c(t)}{T_s}
\end{equation*}
with $T_s$ describing the sampling time, hence
\begin{align*}
    x(k) & := x^c(t_0 + kT_s) \\
    u(k) & := u^c(t_0 + kT_s)
\end{align*}
Then the DT model is given by
\begin{align*}
    x(k+1) & = x(k) + T_s g^c (x(k),u(k)) & = g(x(k),u(k)) \\
    y(k)   & = h^c(x(k),u(k))             & = h(x(k),u(k))
\end{align*}
Therefor a LTI system becomes
\begin{align*}
    x(k+1) & = \overbrace{\mathbb{I} + T_s A^c}^{=A^d} x(k) + \overbrace{T_s B^c}^{=B^d} u(k) \\
    y(k)   & = \underbrace{C^c}_{C^d} x(k) + \underbrace{D^c}_{D^d} u(k)
\end{align*}

\newpar{}
\ptitle{Exact Discretization of LTI}

If the input $u$ is held constant over a sampling interval (ZOH) one can retrieve an exact discretization of the LTI system
\begin{equation*}
    x(t_{k+1}) = \underbrace{e^{A^c T_s}}_{=A} x(t_k) + \underbrace{\int_{0}^{T_s} e^{A^c(T_s - \tau)}B^c d\tau}_{B} u(t_k)
\end{equation*}
with
\begin{equation*}
    B={(A^c)}^{-1}(A-\mathbb{I})B^c
\end{equation*}
if $A^c$ is invertible.

\newpar{}
\ptitle{Solution of DT LTI System}

If the initial state $x(k)$ and the input sequence $\{u(k), \ldots, u(k+N-1)\}$ are known the the solution for the discrete time system at time $k+N$ is given by
\begin{equation*}
    x(k+N) = A^N x(k) + \sum_{i=0}^{N-1} A^i B u(k+N-1-i)
\end{equation*}


\subsection{Linear System Analysis}
\ptitle{DT Stability} $x(k+1) = Ax(k)$ stable iff $\lvert \lambda_j \rvert < 1, \forall j$ \\
\ptitle{LTI DT Controllability} can reach $x^*$ from $x(0)$ in $n$ steps
\begin{align*}
    \mathcal{C} =
    \begin{bmatrix}
        B & \cdots & A^{n-1} B
    \end{bmatrix}
    \ \Rightarrow \ \mathrm{rank}(\mathcal{C}) \overset{!}{=} n
\end{align*}
\ptitle{DT Observability} uniquely distinguish IC from output
\begin{align*}
    \mathcal{O} =
    \begin{bmatrix}
        C^\top & \cdots & {(CA^{n-1})}^\top
    \end{bmatrix}^\top
    \ \Rightarrow \ \mathrm{rank}(\mathcal{O}) \overset{!}{=} n
\end{align*}
\ptitle{Stabilizability} iff all uncontrollable modes are stable
\begin{align*}
    \textrm{if } \mathrm{rank}([\lambda_j \mathbb{I}-A \mid B]) = n \ \forall \lambda_j \in \Lambda_A^+ \ \Rightarrow (A,B) \textrm{ stabilizable}
\end{align*}
\ptitle{Detectablitiy} iff all unobservable modes are stable
\begin{align*}
    \textrm{if } \mathrm{rank}([A^\top - \lambda_j \mathbb{I} \mid C^\top]) = n \ \forall \lambda_j \in \Lambda_A^+ \ \Rightarrow (A,C) \textrm{ detect.}
\end{align*}


\subsection{Nonlinear System Analysis}
\subsubsection[Lyapunov Stability]{Lyapunov Stability (w.r.t \textbf{eq.\ point} $\bar{x}$ of a system)}

\ptitle{Lyapunov Stable}

For every $\epsilon>0$ exists $\delta(\epsilon)$ s.t.\\
\noindent\begin{equation*}
    \lvert\lvert x(0) - \bar{x} \rvert\rvert < \delta(\epsilon) \to
    \lvert\lvert x(k) - \bar{x} \rvert\rvert < \epsilon
\end{equation*}

\paragraph{Globally Asympt. Stable}
Lyapunov stable \& Attractive
\noindent\begin{equation*}
    \lim_{k\to\infty} \lvert\lvert x(k) - \bar{x} \rvert\rvert = 0 \ \forall x(0)
\end{equation*}

\newpar{}
\ptitle{Global Lyapunov Function} (Candidate)

Consider eq point $\bar{x}=0$. $V:\mathbb{R}^n\to \mathbb{R}$, continuous at origin, finite $\forall x$,
\begin{enumerate}
    \item $\lvert\lvert x \rvert\rvert \to \infty \Rightarrow V(x) \to \infty$
    \item $V(0)=0, \quad V(x)>0 \quad \forall x \in \mathbb{R}^n \setminus\{0\}$
    \item $V(g(x)) - V(x) \leq -\alpha(x) \quad \forall x \in \mathbb{R}^n$
\end{enumerate}
where $\alpha:\mathbb{R}^n\to \mathbb{R}$ continuous pos.\ def.

\newpar{}
\ptitle{Global Lyapunov Stability}

If system admits a global Lyapunov function $V(x)$ then $x=0$ is \textbf{Globally Asympt. Stable}

\newpar{}
\textbf{Note} that if $\alpha$ pos.\ \textbf{semi}def $\Rightarrow$ $x=0$ is \textbf{Globally Lyapunov Stable}

\paragraph{Global Lyapunov Stability of Linear Systems}
Given a time-discrete linear system 
\noindent\begin{equation*}
    x(k+1) = Ax(k)
\end{equation*}
The Lyapunov functions 
\noindent\begin{equation*}
    V(x) = x^\top P x, \quad P>0
\end{equation*}
is a global Lyapunov function since it satisfies condition 1,2 per definition and 3 if the \textit{discrete time Lyapunov equation}
\noindent\begin{equation*}
    A^\top PA -P = -Q, \quad Q>0
\end{equation*}
holds.

%Double check if this is correct:
\newpar{}
\ptitle{Closed Loop Control}
When using LQR, the matrix $P$ that satisfies the DARE equation also satisfies the Lyapunov equation
\noindent\begin{equation*}
    {(A+BK)}^\top P{(A+BK)} -P = -Q, \quad Q>0
\end{equation*}
