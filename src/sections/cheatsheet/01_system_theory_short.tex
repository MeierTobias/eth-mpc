\section{System Theory}
\subsection{Discretization}
\ptitle{Euler Discretization (Forward)}
\begin{equation*}
    \dot{x}^c \approx \frac{x^c(t + T_s)-x^c(t)}{T_s}
\end{equation*}
with $T_s$ describing the sampling time, hence
\begin{align*}
    x(k) & := x^c(t_0 + kT_s) \\
    u(k) & := u^c(t_0 + kT_s)
\end{align*}

Therefore a LTI system becomes
\begin{align*}
    x(k+1) & = (\mathbb{I} + T_s A^c) x(k) + T_s B^c u(k) & = A x(k) + B u(k) \\
    y(k)   & = C^c x(k) + D^c u(k)                        & = C x(k) + D u(k)
\end{align*}

\newpar{}
\ptitle{Exact Discretization of LTI (ZOH)}
\begin{equation*}
    x(t_{k+1}) = \underbrace{e^{A^c T_s}}_{=A} x(t_k) + \underbrace{\int_{0}^{T_s} e^{A^c(T_s - \tau)}B^c d\tau}_{B} u(t_k)
\end{equation*}
\begin{equation*}
    B={(A^c)}^{-1}(A-\mathbb{I})B^c
\end{equation*}
if $A^c$ is invertible.
\begin{equation*}
    e^{A^c t} := \sum_{n=0}^{\infty}\frac{{(A^c t)}^n}{n!}
\end{equation*}

\newpar{}
\ptitle{Solution of DT LTI System}
\begin{equation*}
    x(k+N) = A^N x(k) + \sum_{i=0}^{N-1} A^i B u(k+N-1-i)
\end{equation*}

\subsection{LTI DT System Analysis}
\subsubsection{Similarity Transform}
\noindent
\begin{align*}
    \widetilde{x}(k)   & = Tx(k);                              &  & \det(T) \neq 0                                        \\
    \widetilde{x}(k+1) & = TAT^{-1} \widetilde{x}(k) + TB u(k) &  & = \widetilde{A} \widetilde{x}(k) + \widetilde{B} u(k) \\
    y(k)               & = CT^{-1} \widetilde{x}(k) + D u(k)   &  & = \widetilde{C} \widetilde{x}(k) + \widetilde{D} u(k)
\end{align*}

\subsubsection{Controllability and Observability}
\noindent\begin{equation*}
    \mathcal{R} \subseteq \mathcal{C}, \quad \mathcal{R} \subset\mathcal{S}, \quad \mathcal{O} \subset \mathcal{D}
\end{equation*}
% Please ignore the mess I've created here, I don't feel comfortable either.
\begin{center}
    \begin{tikzcd}[ampersand replacement=\&]
        \substack{\text{unstable} \\ \text{modes} \\ \in{} \mathcal{R} \\ \in{} \mathcal{O}}      \&[2em] \substack{\text{all} \\ \text{modes}} \&[-3em] \\[-2em]
        \& \mathcal{C} \arrow[d, teal, xshift=-0.5ex,
        "{\color{teal}\begin{matrix}                        %chktex 18
            \mathsf{CT}: & \text{always}                \\
            \mathsf{DT}: & \mathbf{A}_d \text{ invert.}
        \end{matrix}}
        "'                                                  %chktex 18
        ] \& {\footnotesize\begin{cases}
                    \mathsf{CT:} & \mathbf{x}_0 = 0 \to~\mathbf{x_f} \\
                    \mathsf{DT:} & \mathbf{x}_0 = \mathbf{x_f} \to~0
                \end{cases}}
        \\[2em]
        \mathcal{S}                                     \& \mathcal{R} \arrow[l]\arrow[u, xshift=0.5ex] \&
        {\footnotesize\begin{cases}
                    \mathbf{x}_0 = 0 \to~\mathbf{x_f}
                \end{cases}}
        \\
        \mathcal{D}                                     \& \mathcal{O} \arrow[l]    \&
        {\footnotesize\begin{cases}
                    \mathbf{x}_0 \gets~\mathbf{u},\mathbf{y}
                \end{cases}}
    \end{tikzcd}
\end{center}

\paragraph{Controllability}
Controllable iff
\begin{equation*}
    \mathcal{C} :=
    \begin{bmatrix}
        B & AB & \cdots & A^{n-1} B
    \end{bmatrix}
\end{equation*}
has \textbf{full row rank}.

\newpar{}
\ptitle{Stabilizability}

iff all uncontrollable modes are stable, i.e.\
\begin{itemize}
    \item A system is stabilizable if all unstable modes are controllable.
    \item Controllability always implies stabilizability
\end{itemize}

\ptitle{PBH test}:

\noindent\begin{equation*}
    \text{rank}\left(\left[\lambda_{j}I-A\mid B\right]\right)=n\quad\forall\lambda_{j}\in\Lambda_{A}^{+}\Rightarrow(A,B)\text{ is stabilizable}
\end{equation*}
where $\Lambda_{A}^{+}$ is the set of unstable eigenvalues.

\paragraph{Observability}
Observable iff
\begin{equation*}
    \mathcal{O} :=
    \begin{bmatrix}
        C^{\top}        &
        {(CA)}^{\top}   &
        {(CA^2)}^{\top} &
        \cdots          &
        {(CA^{n-1})}^{\top}
    \end{bmatrix}^{\top}
\end{equation*}
has \textbf{full column rank}.
\newpar{}
\ptitle{Detectability}

iff all unobservable modes are stable
\begin{itemize}
    \item A system is detectable if all unstable modes are observable.
    \item Observability always implies detectability.
\end{itemize}

\newpar{}
\ptitle{PBH test}:

\noindent\begin{equation*}
    \text{rank}\left(\left[A^{\top}-\lambda_{j}I|C^{\top}\right]\right)=n\;\forall\lambda_{j}\in\Lambda_{A}^{+}\Rightarrow (A,C) \text{ is detectable}
\end{equation*}


\subsection{Nonlinear System Analysis}
\subsubsection{Lyapunov Stability}
The equilibrium point $\bar{x}$ of a DT system is \textbf{Lyapunov stable} iff for every $\epsilon >0$ there exists a $\delta(\epsilon)$ such that
\begin{equation*}
    \lVert x(0) - \bar{x} \rVert < \delta(\epsilon) \to \lVert x(k) - \bar{x} \rVert < \epsilon, \; \forall k \geq 0
\end{equation*}

\newpar{}
\ptitle{Global Asymptotic Stability}

An equilibrium point $\bar{x}$ of a system is globally asymptotically stable if it is Lyapunov stable and attractive

\begin{equation*}
    \lim_{k\to\infty} \lVert x(k) - \bar{x} \rVert = 0, \; \forall x(0) \in \mathbb{R}
\end{equation*}

\subsubsection{Global Lyapunov Function}
A function $V:\mathbb{R}^n\to \mathbb{R}$, continuous at the origin, finite $\forall x \in \mathbb{R}^n$ that satisfies
\begin{align*}
    \lVert x \rVert \to \infty \Rightarrow V(x)  \to \infty      &  &  & \text{(radially unbounded)}                                         \\
    V(0)=0 \text{ and } V(x)                     >0              &  &  & \forall x \in \mathbb{R}^n \setminus\{0\}                           \\
    V(g(x)) - V(x)                               \leq -\alpha(x) &  &  & \forall x \in \mathbb{R}^n, \alpha\succ0:\mathbb{R}^n\to \mathbb{R}
\end{align*}
is called a \textbf{Lyapunov function}.

\subsubsection{Local Lyapunov Function}
If the conditions on asymptotical stability only hold on a closed and bounded, positively invariant set $\Omega \subset \mathbb{R}^n$ and $V$ is finite $\forall x \in \Omega$, then the system is locally asymptotically stable in $\Omega$, where $\Omega$ is also called the \textbf{region of attraction (ROA)}.

\subsubsection{Global Lyapunov Stability}
If a system admits a Lyapunov function $V(x)$, then $x=0$ is \textbf{globally asymptotically stable}.
\newpar{}
If $V(x)$ satisfies the conditions only with $\alpha(x)$ being positive \textbf{semi}definite, then $x=0$ is \textbf{globally Lyapunov stable}.

\subsection{Global Lyapunov Stability of LTI DT System}
For a DT LTI system one can construct a Lyapunov function by choosing
\begin{equation*}
    V(x) = x^\top Px,\quad \alpha(x) = x^\top Q x, \;Q \succ 0
\end{equation*}
where $P\succ0$ is the solution of the DT Lyapunov equation
\begin{equation*}
    A^\top PA -P = -Q, \quad Q \succ 0
\end{equation*}
which as a unique solution if $A$ is stable, i.e.\ all eigenvalues inside the unit circle.

\newpar{}
\ptitle{Closed Loop Control}
% TODO: Is this right?
% When using LQR, the matrix $P$ that satisfies the DARE equation also satisfies the Lyapunov equation
\begin{equation*}
    {(A+BK)}^\top P{(A+BK)} -P = -Q, \quad Q\succ 0
\end{equation*}

\textbf{Note} that $Q$ is not the stage cost matrix!

\subsubsection{Direct and Indirect Lyapunov Methods}
\paragraph{Direct Lyapunov Method}
The direct method involves finding a Lyapunov function $V(x)$ and proving that it satisfies the conditions for stability directly.

\paragraph{Indirect Lyapunov Method}
The indirect method uses the linearization and Hartman-Grobman theorem to conclude stability of the nonlinear system from the stability of the linearized system in a local neighborhood of the equilibrium point.