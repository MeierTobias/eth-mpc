\section{Constrained Finite Time Optimal Control (CFTOC)}
\subsection{Constrained Linear Optimal Control}
\noindent
\begin{align*}
    J^*(x(k)) =            & \min_U \: I_f(x_N) + \sum_{i=0}^{N-1}I(x_i,u_i)                    \\
    \text{subject to}\quad & x_{i+1} = Ax_i +Bu_i, \quad i = 0, \ldots, N-1                     \\
                           & x_i \in \mathcal{X}, u_i \in \mathcal{U}, \quad i = 0, \ldots, N-1 \\
                           & x_N \in \mathcal{X}_f                                              \\
                           & x_0 = x(k)
\end{align*}

The feasible set describes a set of initial states for which the optimal control problem is feasible. It is defined only by the constraints and not by the cost:
\begin{align*}
    \mathcal{X}_N = & \left\{ x_0 \in \mathbb{R}^n | \exists\left(u_0, \ldots, u_{N-1}\right) \text{such that } x_i\in\mathcal{X}, u_i\in\mathcal{U} \right. \\
                    & \left.i=0,\ldots, N-1, x_N\in\mathcal{X}_f, \text{where } x_{i+1}=Ax_i + Bu_i\right\}
\end{align*}

\subsubsection{Quadratic Cost CFTOC}
\noindent
\begin{align*}
    J^*(x(k)) =            & \min_U \: x_N^T P x_N + \sum_{i=0}^{N-1}x_i^\top Q x_i + u_i^\top R u_i \\
    \text{subject to}\quad & x_{i+1} = Ax_i +Bu_i, \quad i = 0, \ldots, N-1                          \\
                           & x_i \in \mathcal{X}, u_i \in \mathcal{U}, \quad i = 0, \ldots, N-1      \\
                           & x_N \in \mathcal{X}_f                                                   \\
                           & x_0 = x(k)
\end{align*}
\paragraph{Construction of QP Without Substitution}
\noindent
\begin{align*}
    J^*(x(k)) =            & \min_{z}\begin{bmatrix}
                                         z^\top & x(k)
                                     \end{bmatrix}
    \underbrace{\begin{bmatrix}
                        \bar{H} & 0 \\
                        0       & Q
                    \end{bmatrix}}_{\widetilde{H}}
    \begin{bmatrix}
        z^\top & {x(k)}^\top
    \end{bmatrix}^\top                                       \\
    \text{subject to}\quad & G_{in} z \leq w_{in} +E_{in} x(k) \\
                           & G_{eq} z = E_{eq} x(k)
\end{align*}
\noindent
\begin{equation*}
    z = \begin{bmatrix}
        x_1^\top & \cdots & x_N^\top & u_0^\top & \cdots & u_{N-1}^\top
    \end{bmatrix}^\top \in \mathbb{R}^{n_z := N(n_x + n_u)}
\end{equation*}

\ptitle{Cost}
\begin{equation*}
    \bar{H} = \left[
        \begin{array}{c|c} % ChkTex 44
            \text{blkdiag}(Q, \ldots, Q, P) & 0                            \\
            \hline % ChkTex 44
            0                               & \text{blkdiag}(R, \ldots, R)
        \end{array}
        \right]
\end{equation*}

\ptitle{Equality Constraints}
{\small
    \begin{gather*}
        G_{eq} = \left[
            \begin{array}{cccc|cccc} % ChkTex 44
                I  &   &        &   & -B &    &        &    \\
                -A & I &        &   &    & -B &        &    \\
                   &   & \ddots &   &    &    & \ddots &    \\
                   &   & -A     & I &    &    &        & -B
            \end{array}
            \right] \in \mathbb{R}^{N\cdot n_x \times n_z}\\
        E_{eq} = \begin{bmatrix}
            A^\top & 0 & \cdots & 0
        \end{bmatrix}^\top \in \mathbb{R}^{N\cdot n_x \times n_x}
    \end{gather*}
}

\ptitle{Inequality Constraints}
\begin{align*}
    G_{in} & \in \mathbb{R}^{(n_{in,x}+n_{in,u})\times n_z} \\
           & =\left[
        \begin{array}{c|c} % ChkTex 44
            0                                     & 0                                \\
            \hline % ChkTex 44
            \text{blkdiag}(A_x, \ldots, A_x, A_f) & 0                                \\
            \hline % ChkTex 44
            0                                     & \text{blkdiag}(A_u, \ldots, A_u)
        \end{array}
        \right]
\end{align*}
\begin{align*}
     & w_{in}\in \mathbb{R}^{(n_{in,x}+n_{in,u})\times 1}, \quad E_{in} \in \mathbb{R}^{(n_{in,x}+n_{in,u})\times n_x} \\
     & w_{in} = \begin{bmatrix}
                    {b_x                    }^\top &
                    | {b_x                  }^\top &
                    \cdots                         &
                    {b_x                    }^\top &
                    {b_f                    }^\top &
                    |{b_u }^\top                   &
                    \cdots                         &
                    {b_u                    }^\top
                \end{bmatrix}^\top                                                                       \\
     & E_{in} =\begin{bmatrix}
                   {-A_x}^\top &
                   |0          &
                   \cdots      &
                   0           &
                   0           &
                   |0          &
                   \cdots      &
                   0
               \end{bmatrix}
\end{align*}

\paragraph{Construction of QP With Substitution}
See Subsection~\ref{short:sec:batch_lqr}
\begin{equation*}
    X = S^x x(k) + S^u U
\end{equation*}

%TODO: what exactly is $Y$  and does it matter as x(k) is constant anyway?
\ptitle{Cost}
\begin{align*}
    J^*(x(k), U)=          & \min_{U}\begin{bmatrix}
                                         U^\top & {x(k)}^\top
                                     \end{bmatrix}
    \underbrace{\begin{bmatrix}
                        H & F^\top \\
                        F & Y
                    \end{bmatrix}}_{\widetilde{H}}
    {\begin{bmatrix}
         U^\top & {x(k)}^\top
     \end{bmatrix}}^\top                                 \\
    \text{subject to}\quad & GU \leq w + Ex(k)
\end{align*}

\ptitle{Inequality Constraints}
\begin{align*}
    G & \in \mathbb{R}^{(n_{in,u} + n_{in,x})\times (N\cdot n_u)} \\
      & = \begin{bmatrix}
              A_u                   & 0             & \cdots & 0      \\
              0                     & A_u           & \cdots & 0      \\
              \vdots                & \vdots        & \ddots & \vdots \\
              0                     & 0             & \cdots & A_u    \\
              \cmidrule(lr){1-4}  0 & 0             & \cdots & 0      \\
              A_x B                 & 0             & \cdots & 0      \\
              A_x AB                & A_x B         & \cdots & 0      \\
              \vdots                & \vdots        & \ddots & \vdots \\
              A_f A^{N-1} B         & A_f A^{N-2} B & \cdots & A_f B
          \end{bmatrix}
\end{align*}
\begin{gather*}
    w \in \mathbb{R}^{(n_{in,u} + n_{in,x})\times 1},\quad E \in \mathbb{R}^{(n_{in,u} + n_{in,x})\times n_x}\\
    w = \begin{bmatrix}
        {b_u                    }^\top &
        \cdots                         &
        {b_u                    }^\top &
        |{b_x                   }^\top &
        {b_x                    }^\top &
        \cdots                         &
        {b_f                    }^\top
    \end{bmatrix}, \\
    E = \begin{bmatrix}
        {0                       }^\top &
        \cdots                          &
        {0                       }^\top &
        |{ -A_x                  }^\top &
        {-A_x A                  }^\top &
        \cdots                          &
        {-A_f A^N}^\top
    \end{bmatrix}
\end{gather*}

\paragraph{Comparison: Substitution vs.\ No Substitution}

\renewcommand{\arraystretch}{1.3}
\setlength{\oldtabcolsep}{\tabcolsep}\setlength\tabcolsep{6pt}
\begin{tabularx}{\linewidth}{@{}lll@{}}
                   & Substitution                  & No Substitution               \\
    \cmidrule{2-3}
    \# opt.\ vars. & $N n_u$                       & $N(n_x + n_u)$                \\
    benefits       & less opt.\ vars.              &                               \\
                   & less constraints              & sparse constr.\ ($\propto N$) \\
    drawbacks      & complicated constr.           & more opt.\ vars.              \\
                   & more constr.\ ($\propto N^2$) &                               % TODO: Why?
\end{tabularx}
\renewcommand{\arraystretch}{1}
\setlength\tabcolsep{\oldtabcolsep}
Note that substitution transforms input constraints into state constraints, making them more involved in general. Hence, no substitution is often preferable. For small $N$ and large $n_x$ however, substitution can be more efficient.

\paragraph{State Feedback Solution}

As $n_x>=1$, the CFTOC problem is a \textbf{multiparametric quadratic program (mp-QP)} in general with the following solution properties:

\begin{itemize}
    \item $u_0^*$ is of the form (nonlinear feedback policy)
          \begin{equation*}
              u_0^*=\kappa(x(k)),\quad\forall x(k)\in\mathcal{X}_0
          \end{equation*}
          with $\kappa:\mathbb{R}^n \to \mathbb{R}^m$ cont., \textbf{piecewise affine} on polyhedra
          \begin{equation*}
              \kappa  =\kappa(x)=F^j x + g^j, \quad\mathrm{if}\quad x\in CR^j,\quad j=1,\cdots,N^r
          \end{equation*}
    \item The polyhedral sets for the individual control laws
          \begin{equation*}
              CR^j=\{x\in\mathbb{R}^n|H^j x\leq K^j\},j=1,\cdots,N^r
          \end{equation*}
          are a partition of the feasible polyhedron $\mathcal{X}_0$.
    \item $J^*(x(k))$ is \textbf{convex}, \textbf{piecewise quadratic} on polyhedra.
\end{itemize}
\newpar{}
\textbf{Explicit MPC} addresses how to compute this solution.

\subsubsection[1-Norm and Inf-Norm Cost]{1-Norm and $\infty$-Norm Cost CFTOC}
\paragraph[linf Minimization]{$l_{\infty}$ Minimization}
\noindent
\begin{gather*}
    \min_{x \in \mathbb{R}^n} \|x\|_\infty = \min_{x \in \mathbb{R}^n} \left[\max \{x_1, \dots, x_n, -x_1, \dots, -x_n\}\right] \\
    \text{subj.\ to } Fx \leq g
\end{gather*}

\newpar{}
\ptitle{Auxiliary Variable Formulation}
\begin{gather*}
    \min_{x, t} t \\
    \text{subj.\ to } -\mathbf{1} t \leq x \leq \mathbf{1} t\\
    Fx \leq g
\end{gather*}

\ptitle{Application to CFTOC}

Introduces a \textit{scalar} auxiliary variable for each state of the trajectory
\begin{align*}
    z & := \{ \varepsilon_{0}^x, \dots, \varepsilon_{N}^x, \varepsilon_{0}^u, \dots, \varepsilon_{N-1}^u, u_0^\top, \dots, u_{N-1}^\top \} \in \mathbb{R}^s, \\
    s & := (m + 1) N + N + 1
\end{align*}
\noindent
\begin{align*}
    \min_z\quad          & \varepsilon_{0}^x + \dots + \varepsilon_{N}^x + \varepsilon_{0}^u + \dots + \varepsilon_{N-1}^u \\
    \text{subj.\ to } \  & -\mathbf{1}_n \varepsilon_{x_i} \leq  Q x_i \leq \mathbf{1}_n \varepsilon_{x_i}                 \\
                         & -\mathbf{1}_r \varepsilon_{x_N} \leq  P x_N \leq \mathbf{1}_r \varepsilon_{x_N}                 \\
                         & -\mathbf{1}_m \varepsilon_{u_i} \leq  R u_i \leq \mathbf{1}_m \varepsilon_{u_i}                 \\
                         & x_i = A^i x_0 + \sum_{j=0}^{i-1} A^j B u_{i-1-j}\in \mathcal{X}                                 \\
                         & u_i \in \mathcal{U}, x_N \in \mathcal{X}_f, x_0 = x(k)
\end{align*}

\paragraph[l1 Minimization]{$l_{1}$ Minimization}
\noindent
\begin{gather*}
    \min_{x \in \mathbb{R}^n} \|x\|_1 = \min_{x \in \mathbb{R}^n} \left[ \sum_{i=1}^{m} \max \{x_i, -x_i\} \right] \\
    \text{subj.\ to } Fx \leq g
\end{gather*}

\ptitle{Auxiliary Variable Formulation}
\noindent
\begin{align*}
    \min_{x \in \mathbb{R}^n, \mathbf{t} \in \mathbb{R}^n}\quad & \mathbf{1}^\top \mathbf{t},         \\
    \text{subj.\ to }\quad                                      & -\mathbf{t} \leq x \leq \mathbf{t}, \\
                                                                & Fx \leq g
\end{align*}

\ptitle{Application to CFTOC}
\noindent
\begin{align*}
    z & := \{ {(\bm{\varepsilon}_{0}^x)}^\top, \dots, {(\bm{\varepsilon}_{N}^x)}^\top, {(\bm{\varepsilon}_{0}^u)}^\top, \dots, {(\bm{\varepsilon}_{N-1}^u)}^\top, u_0^\top, \dots, u_{N-1}^\top \} \\
    z & \in \mathbb{R}^s \text{ with } s := n(N+1) + 2mN
\end{align*}
\begin{align*}
    \min_z \quad            & \mathbf{1}^\top \bm{\varepsilon}_{0}^x + \dots + \mathbf{1}^\top \bm{\varepsilon}_{N}^x + \mathbf{1}^\top \bm{\varepsilon}_{0}^u + \dots + \mathbf{1}^\top \bm{\varepsilon}_{N-1}^u \\
    \text{subj.\ to } \quad & -\bm{\varepsilon}_{x_i} \leq Q x_i \leq\bm{\varepsilon}_{x_i}                                                                                                                       \\
                            & -\bm{\varepsilon}_{x_N} \leq P x_N \leq\bm{\varepsilon}_{x_N}                                                                                                                       \\
                            & -\bm{\varepsilon}_{u_i} \leq R u_i \leq\bm{\varepsilon}_{u_i}\end{align*}

\newpar{}
\ptitle{General Formulation}
\begin{equation*}
    y = \begin{bmatrix}
        x \\\mathbf{z}
    \end{bmatrix}\in \mathbb{R}^{(n+N)}
\end{equation*}
\begin{align*}
    \min_x \|Ax\|_1 &  &  & \Leftrightarrow & \min_y       & \begin{bmatrix}
                                                                 0 & \mathbf{1}^\top
                                                             \end{bmatrix}y \\
                    &  &  &                 & \text{s.t. } & \begin{bmatrix}
                                                                 A  & -\mathbb{I} \\
                                                                 -A & -\mathbb{I}
                                                             \end{bmatrix}y\leq0
\end{align*}

\paragraph[l1, linf State Feedback Solution]{$l_{1}, l_{\infty}$ State Feedback Solution}
\noindent
\begin{align*}
    \min_{z\in\mathbb{R}^n}c^T z                            \\
    \text{subj.\ to } \bar{G}z & \leq \bar{w} + \bar{S}x(k)
\end{align*}
is again a mp-LP with the following solution properties:
\begin{itemize}
    \item $u_0^*$ has the form:
          \begin{align*}
              u_0^* = \kappa(x(0)), \quad \forall x(0) \in \mathcal{X}_0,
          \end{align*}
          where $\kappa : \mathbb{R}^n \to \mathbb{R}^m$ is cont.\, piecewise affine on polyhedra:
          \begin{align*}
              \kappa(x) = F^j x + g^j, \quad \text{if } x \in CR^j, \quad j = 1, \dots, N^r
          \end{align*}
    \item The polyhedral sets
          \begin{equation*}
              CR^j = \{x \in \mathbb{R}^n \mid H^j x \leq K^j\}, \quad j = 1, \dots, N^r
          \end{equation*}
          are a partition of the feasible polyhedron $\mathcal{X}_0$.
    \item In case of multiple optimizers, a \textbf{piecewise affine} control law exists.
    \item The value function $J^*(x(0))$ is \textbf{convex} and \textbf{piecewise affine} on polyhedra.
\end{itemize}

\subsection{Common Constraints}
\subsubsection{Polytopic Constraints}
\ptitle{Input Constraints}
\begin{equation*}
    u_{\min} \leq u \leq u_{\max} \quad \Leftrightarrow \quad \begin{bmatrix}
        -\mathbb{I} \\
        \mathbb{I}
    \end{bmatrix} u \leq \begin{bmatrix}
        -u_{\min} \\
        u_{\max}
    \end{bmatrix}
\end{equation*}

\newpar{}
\ptitle{Rate Constraints}
\begin{equation*}
    \|x_K - x_{k+1}\|_\infty \leq \alpha \quad \Leftrightarrow \quad \begin{bmatrix}
        \mathbb{I}  & -\mathbb{I} \\
        -\mathbb{I} & \mathbb{I}
    \end{bmatrix}\begin{bmatrix}
        x_k \\
        x_{k+1}
    \end{bmatrix} \leq \bm{1} \alpha
\end{equation*}

\newpar{}
\ptitle{Magnitude Constraints}

\begin{equation*}
    \|C x_k\|_\infty \leq \alpha \quad \Leftrightarrow \quad \begin{bmatrix}
        C \\ -C
    \end{bmatrix} x_k \leq \bm{1} \alpha
\end{equation*}