\section{Feasibility and Stability}
Using LQR in constrained systems can e.g.\ lead to input saturation and instability. However, even MPC does not guarantee feasibility and stability per se. In particular, in finite horizon MPC
\begin{itemize}
    \item Decrease in the prediction horizon can cause loss of stability properties.
    \item Depending on the initial condition, the CL trajectory may lead to states for which the optimization problem is infeasible after some steps, even without disturbance or model mismatch.
\end{itemize}
If we (could) solve the $\infty$-horizon control problem:
\begin{itemize}
    \item problem is feasible $\rightarrow$ closed loop trajectories will be always feasible.
    \item the cost is finite $\rightarrow$ states and inputs will converge asymptotically to the origin.
\end{itemize}
i.e.\ OL and CL would be identical.
\newpar{}
The goal of this section is hence to approximate the infinite horizon MPC to get feasibility and stability guarantees.
\subsection{Zero Terminal Constraint}
Given the terminal constraint $x_N = 0$.
\subsubsection{Feasibility}
Assume feasibility for $x(k)$ with optimal solution
\begin{equation*}
    \{u_0^*, u_1^*, \ldots, u_{N-1}^*\}, \quad \{x(k),x_1^*, x_2^*, \ldots, x_N^*\}
\end{equation*}

Applying the first control input $u_0^*$ to the system, the state will evolve as
\begin{equation*}
    x(k+1) = Ax(k) + Bu(k) = Ax(k) + B u_0^* = x_1^*
\end{equation*}
it follows directly that the sequence
\begin{equation*}
    \widetilde{U}=\{u_1^*, \ldots, u_{N-1}^*, 0\}\; \to \; \widetilde{X}=\{x_1^*, \ldots, x_N^*, \underbrace{Ax_N^* + Bu_N}_{0}\}
\end{equation*}
is feasible too. And therefore, recursive feasibility is guaranteed and the feasible set is \textbf{control invariant}.

\subsubsection{Stability}
Using the same (suboptimal) sequence $\widetilde{x}$ as above, the cost is given by
\begin{align*}
    J^*(x(k+1)) & \leq \widetilde{J}(x(k+1))                                                                                                         \\
                & = \sum_{i=1}^{N-1} I(x_i^*, u_i^*) + I(x_N^*, 0)                                                                                   \\
                & =\underbrace{\sum_{i=0}^{N-1} I(x_i^*, u_i^*)}_{J^*(x(k))} - \underbrace{I(x_0^*, u_0^*)}_{\geq 0}  + \underbrace{I(x_N^*, 0)}_{0} \\
                & \leq J^*(x(k))
\end{align*}
\begin{center}
    \includegraphics[width=0.8\linewidth]{images/06_fin_hor_stab.png}
\end{center}

\subsection{Terminal Subset Constraint}
Problem: The terminal constraint $x_N = 0$ reduces the size of the feasible set. Goal: Use convex set $\mathcal{X}_f$ to increase the region of attraction.
% TODO once recording online: Add proof that tail cost (from j to inf) <= terminal cost (didn't find it in slides but was in lecture).
\subsubsection{Feasibility}

\subsubsection{Stability}