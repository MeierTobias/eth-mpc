\section{Invariance}
\ptitle{Limitations of Linear Controllers}

The region in which a linear controller \textit{never} violates state and input constraints is very limited.
% TODO: I think the regions represent the feasible set, right? I.e. the x1 and x2 axis are x_{1,0} and x_{2,0}. If true, uncomment the following ;)
% The range of initial conditions for which a linear controller \textit{never} violates state and input constraints is very limited.
\begin{center}
    \includegraphics[width = 0.5\linewidth]{05_LQR_limitations.png}
\end{center}
Nonlinear control (MPC) can be used to increase the region, for which constraints can always be satisfied.

\subsection{Invariance}
The concept of invariance is used to access constraint satisfaction, for
\begin{itemize}
    \item an autonomous system $x(k + 1) = g(x(k))$
    \item or a CL system $x(k + 1) = g(x(k), \kappa(x(k)))$, given a controller $\kappa$.
\end{itemize}

\subsubsection{Invariant Sets}
\ptitle{Positively Invariant Set}

A set $\mathcal{O}$ is said to be a positively invariant set for an autonomous system if
\begin{equation*}
    x(k) \in \mathcal{O} \Rightarrow x(k + 1) \in \mathcal{O}, \quad \forall k \in \{0, 1, \dots \}
\end{equation*}
Hence, if $\mathcal{O}$ is within the constraints, it provides a set of initial states from which the trajectory will \textbf{never} violate the system constraints.

\newpar{}
\ptitle{Maximal Positively Invariant Set $\mathcal{O}_\infty$}

The set $\mathcal{O}_\infty \subset \mathcal{X}$ is the maximal positively invariant set with respect to $\mathcal{X}$ if $\mathcal{O}_\infty$ is positively invariant and $\mathcal{O}_\infty$ contains all positively invariant sets.
\newpar{}
$\mathcal{O}_\infty$ is the set of \textbf{all} states for which the system will remain feasible once in $\mathcal{O}_\infty$.

\newpar{}
\ptitle{Geometric Condition for Invariance}

A set $\mathcal{O}$ is a positively invariant set if and only if
\begin{equation*}
    \mathcal{O} \subseteq \text{pre}(\mathcal{O})
\end{equation*}
Note that
\begin{equation*}
    \mathcal{O} \subseteq \text{pre}(\mathcal{O}) \iff \text{pre}(\mathcal{O})\cap \mathcal{O} = \mathcal{O}
\end{equation*}
\subsubsection{Pre-Sets}

Given a set $S$ and the dynamic system $x(k + 1) = g(x(k))$, the pre-set of $S$ is the set of states that evolve into the target set $S$ in \textbf{one} time step:
\begin{equation*}
    \text{pre}(S) := \{x \mid g(x) \in S \}
\end{equation*}

\newpar{}
\ptitle{Linear Autonomous Systems}

Given $x(k + 1) = A x(k)$, the condition is
\begin{equation*}
    \text{pre}(S) := \{x \mid A x \in S \}
\end{equation*}
In the constraint case $S := \{x \mid F x \leq f \}$, this yields
\begin{equation*}
    \text{pre}(S) = \{x \mid F A x \leq f \}
\end{equation*}

\subsubsection{Computing Invariant Sets}\label{ssec:computing_invariant_sets}
A conceptual algorithm to calculate invariant sets is
\begin{algorithmic}
    \State{} \textbf{Input:} $g$, $X$
    \State{} \textbf{Output:} $\mathcal{O}_\infty$
    \State{} $\Omega_0 \gets X$
    \While{true}
    \State{} $\Omega_{i+1} \gets \text{pre}(\Omega_i) \cap \Omega_i$
    \If{$\Omega_{i+1} = \Omega_i$}
    \State{} \textbf{return} $\mathcal{O}_\infty = \Omega_i$
    \EndIf{}
    \EndWhile{}
\end{algorithmic}
This algorithm generates the set sequence $\{\Omega_i\}$ satisfying $\Omega_{i+1} \subseteq \Omega_i$ for all $i \in \mathbb{N}$ and terminates when $\Omega_{i+1} = \Omega_i$, where $\Omega_i$ is the maximal positively invariant set $\mathcal{O}_\infty$ for $x(k + 1) = g(x(k))$.

\subsection{Control Invariance}
\subsubsection{Control Invariant Sets}
A set $\mathcal{C} \subseteq \mathcal{X}$ is said to be a control invariant set if
\begin{gather*}
    x(k) \in \mathcal{C} \Rightarrow \exists u(k) \in \mathcal{U}\\
    \text{such that } g(x(k), u(k)) \in \mathcal{C} \text{ for all } k \in \mathbb{N}^+
\end{gather*}
If this set is feasible (wrt.\ constraints) then we can be sure that the constraints will be satisfied for all time under the given control law.

\newpar{}
\ptitle{Maximal Control Invariant Set $\mathcal{C}_\infty$}

The set $\mathcal{C}_\infty$ is said to be the maximal control invariant set for the system $x(k + 1) = g(x(k), u(k))$ subject to the constraints $(x, u) \in \mathcal{X} \times \mathcal{U}$ if it is control invariant and \textbf{contains all control invariant sets} contained in $\mathcal{X}$.
\newpar{}
For all states contained in the maximal control invariant set $\mathcal{C}_\infty$, there exists a control law such that the system constraints are never violated.

\subsubsection{Conceptual Calculation of Control Invariant Sets}

The concept of a pre-set extends to systems with exogenous inputs:
\begin{equation*}
    \text{pre}(S) := \{x \mid \exists u \in \mathcal{U} \text{ s.t. } g(x, u) \in S \}
\end{equation*}
A set $\mathcal{C}$ is a control invariant set if and only if $\mathcal{C} \subseteq \text{pre}(\mathcal{C})$.

\newpar{}
The same algorithm as for positively invariant sets can be used but the calculation of the pre-set is much more complicated.


\ptitle{Preset Computation for Constrained LTI System}

Consider the system $x(k + 1) = A x(k) + B u(k)$ under the constraints
\begin{align*}
    u(k) \in \mathcal{U} & := \{u \mid G u \leq g \} \\
    S                    & := \{x \mid F x \leq f \}
\end{align*}
The preset can be computed as
\begin{align*}
    \text{pre}(S) & = \{x \mid \exists u \in \mathcal{U}, \; Ax + Bu \in S \}                                                                                                            \\
                  & = \{x \mid \exists u \in \mathcal{U}, \; F A x + F B u \leq f \}                                                                                                     \\
                  & = \left\{ x \mid \exists u, \begin{bmatrix} F A & F B \\ 0 & G \end{bmatrix} \begin{bmatrix} x \\ u \end{bmatrix} \leq \begin{bmatrix} f \\ g \end{bmatrix} \right\}
\end{align*}
which is a \textbf{projection operation}.

\newpar{}
\ptitle{Properties}

\begin{itemize}
    \item An entire set of states can map into each point
    \item The pre-set is a lot larger (positive), but much more difficult to compute
    \item The maximum control invariant set is the best any controller can do
\end{itemize}

\paragraph{Control Laws from Control Invariant Sets}

Let $\mathcal{C}$ be a control invariant set for the system $x(k + 1) = g(x(k), u(k))$.
\newpar{}
A control law $\kappa(x(k))$ guarantees that the system $x(k + 1) = g(x(k), \kappa(x(k)))$ will satisfy the state constraints for all time if it respects the control invariant set, i.e.:
\begin{equation*}
    g(x, \kappa(x)) \in \mathcal{C} \quad \forall x \in \mathcal{C}
\end{equation*}
Therefore, we can synthesize a control law from a control invariant set by solving the following \textbf{optimization problem}:
\begin{equation*}
    \kappa(x) := \arg\min \{ f(x, u) \mid g(x, u) \in \mathcal{C},\; u \in \mathcal{U} \},
\end{equation*}
where $f$ is any function (including $f(x, u) = 0$).
\newpar{}
\textbf{Note}: This ensures that the system will satisfy the constraints, but it does not guarantee convergence.

\subsection{Computation of Simple Invariant Sets}
This subsection specifies how to compute the intersection of two sets and the equality test needed in~\ref{ssec:computing_invariant_sets}.
\subsubsection{Polytopes}
\paragraph{Intersection}
\begin{equation*}
    S \cap T = \left\{x \Bigg|\begin{bmatrix}
        C \\D
    \end{bmatrix}x \leq \begin{bmatrix}
        c \\d
    \end{bmatrix}\right\}
\end{equation*}

\paragraph{Equality Test (Subset Test)}
$P=\{x\mid Cx\leq c\}$ is a subset of $Q=\{x\mid Dx\leq d\}$ if for each row, the \textbf{support} is a subset of $D$:
\begin{equation*}
    h_P(D_i) \leq d_i
\end{equation*}
where the support (extremum of $P$ in direction $D_i$) is defined as
\begin{align*}
    h_p(D_i) :=     & \max_x D_i x \\
    \mathrm{s.t.}\; & Cx\leq c
\end{align*}

\subsubsection{Ellipsoids}
Invariant ellipsoidal sets with its center in the origin, can directly be computed from Lyapunov functions.

\newpar{}
If $V:\mathbb{R}^d\to\mathbb{R}$ is a Lyapunov function for the system $x(k+1)=g(x(k))$, then the sublevel set
\begin{equation*}
    Y = \{x\mid V(x)\leq \alpha\}
\end{equation*}
is an invariant set for all $\alpha \geq 0$.

\paragraph{Linear Systems}
For linear systems $x(k+1)=Ax(k)$, the Lyapunov function $V(x)=x^\top Px$ with $P\succ 0$  and
\begin{equation*}
    A^\top PA-P \prec 0
\end{equation*}
yields the ellipsoidal invariant set

\begin{equation*}
    Y_\alpha = \left\{x\mid x^\top Px\leq \alpha\right\} \subset \mathcal{X}= \left\{x\mid Fx\leq f\right\}
\end{equation*}

\newpar{}
\ptitle{Maxmium Invariant Set}

The largest ellipse $Y_\alpha$ in a polytope $\mathcal{X}$ can be computed with
\begin{equation*}
    \alpha^* = \min_{i\in \{1,\ldots, n\}} \frac{f_i^2}{F_i P^{-1} F_i^\top}
\end{equation*}

\textbf{Note} that this only computes the largest ellispoid with the center in the origin and a fixed shape $P$.